% Dies ist die Preambel
% \documentclass[12pt,a4paper,bibtotoc, twoside, cleardoublepage=empty]{article} % twoside
\documentclass[12pt,a4paper,bibtotoc, cleardoublepage=empty]{article}  % oneside centered
\bibliographystyle{plain}

\usepackage{pdfpages} % import pdf


% das ist Standard - nie ohne aus dem Haus gehen
\usepackage[german]{babel}
% Biber!!!!!!!!!!!!!!!!!!


\usepackage{ebgaramond}

\usepackage{typearea}
\usepackage[style=authortitle]{biblatex}
\usepackage[babel,german=guillemets]{csquotes}
\bibliography{lit/literatur}



\usepackage[onehalfspacing]{setspace}

\usepackage{tabularx} % in the preamble

% Fu�noten auch in �berschriften
\usepackage[stable]{footmisc}

% Bilder
\usepackage{graphicx}
\graphicspath{ {../img/} } 
\usepackage{float} 

% Seitenraender 
\usepackage{geometry}

%\geometry{a4paper, top=20mm, left=35mm, right=20mm, bottom=20mm,headsep=12mm, footskip=12mm} % twoside
\geometry{a4paper, top=20mm, left=20mm, right=20mm, bottom=20mm,headsep=12mm, footskip=12mm} % oneside centered


%\usepackage{natbib}


%Hinweisbox
\usepackage{calc}
\usepackage{hhline} 
\usepackage{multirow} 
\usepackage{xcolor}
\usepackage{colortbl}
\usepackage{graphicx}

\newlength{\iconwidth}
\setlength{\iconwidth}{1cm}

\definecolor{boxheadcol}{gray}{.6}
\definecolor{boxcol}{gray}{.9}

\newenvironment{displaybox}[2]{%
  \begin{center}
    \setlength\arrayrulewidth{0.75pt}%
    \arrayrulecolor{white}%
    \renewcommand{\arraystretch}{1.3}%
    \begin{tabular}{p{\iconwidth}p{\linewidth-4\tabcolsep-\iconwidth}}
      \multirow{2}{*}{#2}&\cellcolor{boxheadcol}\textbf{\sffamily\color{white}#1} \\%
      \hhline{~-}%
      &\cellcolor{boxcol}%
}{%
      \\
    \end{tabular}
  \end{center}%
}


\newenvironment{Tipp}{%
\begin{displaybox}{Tipp}{\includegraphics[width=\iconwidth]{img/com/icon-tipp}}}%
{\end{displaybox}}

\newenvironment{Hinweis}{%
\begin{displaybox}{Hinweis}{\includegraphics[width=\iconwidth]{img/com/icon-hinweis}}}%
{\end{displaybox}}
%Hinweisbox ende

% Standart Kopfzeile 
\pagestyle{headings}
  
% Referenzen

\usepackage{hyperref}
\hypersetup{
  colorlinks=true,
  linkcolor=black,
	citecolor=black,
  urlcolor=blue,
  pdfborder={0 0 0}
}

% Mit Mausklick zum Ziel
\usepackage{nameref}
% URLs
\usepackage{url} 



% Umlaute:
% Immer nur einen inputenc verwenden, sonst Fehler!
% Linux
% \usepackage[latin1]{inputenc} 
% Windows
\usepackage[utf8]{inputenc}

% Umlaute auch in der PDF
\usepackage[T1]{fontenc}

% Fuer jede Section eine neue Seite
\let\stdsection\section
\renewcommand\section{\newpage\stdsection}

% Fuer jede SubSection eine neue Seite
%\let\stdsubsection\subsection
%\renewcommand\subsection{\newpage\stdsubsection}

%\usepackage{natbib}	% Literaturverzeichnis

% \usepackage{skull}	% alles hat ein Ende

\usepackage{color}	% bring Farbe ins Spiel
% Fuer Codebeispiele
\definecolor{DarkPurple}{rgb}{0.4,0.1,0.4}
\definecolor{DarkCyan}{rgb}{0.0,0.5,0.4}
\definecolor{LightLime}{rgb}{0.4,0.6,0.5}
\definecolor{Blue}{rgb}{0.0,0.0,1.0}

\definecolor{forestgreen}{RGB}{34,139,34}
\definecolor{orangered}{RGB}{239,134,64}
\definecolor{darkblue}{rgb}{0.0,0.0,0.6}
\definecolor{gray}{rgb}{0.4,0.4,0.4}

% sch?nere Serifenfonts
\usepackage{times}		
\usepackage{lmodern}
	
% deutsche Abs?tze
\parskip2ex		% Absatzabsstand	
\parindent0ex		% Absatzeinzug

% keine Hurenkinder und Schusterjungen
\clubpenalty=10000
\widowpenalty=10000

% Fuer mehr Codeschnipsel Funktionen
\usepackage{moreverb}

\usepackage{listings}

% f?r Java-Bezeichner und -Keywords im Flie?text
\newcommand{\code}[1]{\small\lstinline[style=InlineJava]!#1!\normalsize}
%\newcommand{\code}[1]{\scriptsize\texttt{#1}\normalsize}

% fuer Listings mit Eintrag im Inhaltsverzeichnis
%\newcommand{\newlisting}[2]{
%\subsubsection*{Listing \ref{lst:#1}: #2}
%\addcontentsline{toc}{subsubsection}{\ref{lst:#1}. #2}}

\let\underscore\_
\newcommand{\myunderscore}{\renewcommand{\_}{\underscore\hspace{0pt}}}
%Issue the changed underscore command to the whole document.
\myunderscore

\lstdefinestyle{Java}
{
language=Java,
numberfirstline,
numberstyle=\tiny\sffamily,
tabsize=5,
captionpos=b,
aboveskip=1em,
belowskip=1em,
columns=flexible,
xleftmargin=2em,
xrightmargin=1em,
frame=single,
frameround=tttt,
commentstyle=\itshape\color{LightLime},
keywordstyle=\bfseries\color{DarkPurple},
basicstyle=\footnotesize\ttfamily,
stringstyle=\color{Blue},
showstringspaces=false,
}

\lstdefinestyle{XML} {
    language=XML,
    extendedchars=true, 
    breaklines=true,
    breakatwhitespace=true,
    emph={},
    emphstyle=\color{red},
    basicstyle=\ttfamily,
    columns=fullflexible,
    commentstyle=\color{gray}\upshape,
    morestring=[b]",
    morecomment=[s]{<?}{?>},
    morecomment=[s][\color{forestgreen}]{<!--}{-->},
    keywordstyle=\color{orangered},
    stringstyle=\ttfamily\color{black}\normalfont,
    tagstyle=\color{darkblue}\bf,
    morekeywords={attribute,xmlns,version,type,release},
}




\lstnewenvironment{javalisting}[1][]
{
	\lstset{language=Java, 
					 style=Java
	}
}
{}

\newenvironment{javalistingfigure}
{
\begin{figure}
\begin{javalisting}
}
{
\end{javalisting}
	\caption{asasas}
	\label{fig:javalisting}
\end{figure}
}


%\lstnewenvironment{javalisting}
%{
%\begin{center}
	%\begin{figure}

%		\begin{lstlisting}[style=Java]
		
		%		public class UserSession implements Serializable{}

%}
%{
%		\end{lstlisting} 

		%\caption{dsdsds}
		%\label{fig:sddsdsds}
		
	%\end{figure}
%\end{center}
%}

% Strikeout
\usepackage{ulem}

% Zeilenumbruch Bib
\renewcommand*{\labelnamepunct}{\newunitpunct\par}